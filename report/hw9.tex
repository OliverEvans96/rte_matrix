\documentclass[10pt]{article}
\usepackage{geometry}
\usepackage{mathtools}
\usepackage{amssymb}
\usepackage{enumerate}
\usepackage{systeme}
\usepackage{listings}
\usepackage{empheq}
\usepackage{float}
\usepackage{tikz}

% Set latex arrowheads by default
\tikzset{>=latex}

\newcommand\mgin{1in}
\geometry{
	left=\mgin,
	right=\mgin,
	bottom=\mgin,
	top=\mgin
}

% Natural numbers
\newcommand\N{\mathbb{N}}
% Real numbers
\newcommand\R{\mathbb{R}}
% Complex numbers
\newcommand\C{\mathbb{C}}
% Curly B for basis
\newcommand\BB{\mathcal{B}}
% Curly R for range (not real numbers)
\newcommand\RR{\mathcal{R}}
% Curly N for null space
\newcommand\NN{\mathcal{N}}
% Norm
\newcommand\norm[1]{\left\lVert #1 \right\rVert}
% Uniform Norm
\newcommand\unorm[1]{\left\lVert #1 \right\rVert_\infty}
% Inner Product
\newcommand\ip[1]{\left\langle #1 \right\rangle}
% Absolute value
\newcommand\abs[1]{\left| #1 \right|}
% Complex Conjugate
\newcommand\conj\overline
% Disable paragraph indentation
\setlength{\parindent}{0pt}
% Space after question
\newcommand\qspace{-0.1in}
% Line after question
\newcommand\ql{\centerline{\rule{3in}{0.5pt}}\vspace{1em}}
% End of proof
\newcommand\qed{\hfill$\blacksquare$\hspace{0.5in}}

% Number this equation
\newcommand\eqnum{\addtocounter{equation}{1}\tag{\theequation}}

% Section labels
\renewcommand\thesubsection{(\alph{subsection})}

% Code input settings
\lstset{
	basicstyle=\ttfamily,
	commentstyle=\rmfamily,
	showstringspaces=false,
	numbers=left, 
	breaklines=true,
	numbersep=1em,
	frame=leftline,
	tabsize=4,
	xleftmargin=.5in,
	xrightmargin=.5in} 

% arara: pdflatex
\begin{document}

%%fakesection Title
\null

\thispagestyle{empty}
\addtocounter{page}{-1}

\begin{center}
    \begin{sffamily}
	\begin{bfseries}
	    \null
	    \vfill
	    \Huge{Methods of Applied Math I} \\

	    \vspace{10pt}
		\huge{Homework 9} \\
	    \vspace{20pt}
    \begin{LARGE}
		Oliver Evans \\
		Dr. Pat Wilber \\[.5em]
	\vspace{20pt}
	\today
    \end{LARGE}
	\end{bfseries}
    \end{sffamily}
    \vspace{30pt}

    \null
    \vfill
    \vfill
    \null
\end{center}
\pagebreak

\section{}
Consider the boundary-value problem
\begin{equation}
	Lu = f, \quad B_1u = \gamma_1, \quad B_2u = \gamma_2
\end{equation}

where
\begin{equation}
	Lu = a_2(x)\frac{d^2u}{dx^2} + a_1(x)\frac{du}{dx} + a_0(x) u \mbox{ for } a<x<b
\end{equation}
\begin{equation}
	B_1u = \alpha_{11}u(a) + \alpha_{12}\frac{du}{dx}(a) \beta_{11}u(b) + \beta_{12}\frac{du}{dx}(b),
\end{equation}
\begin{equation}
	B_2u = \alpha_{21}u(a) + \alpha_{22}\frac{du}{dx}(a) \beta_{21}u(b) + \beta_{22}\frac{du}{dx}(b),
\end{equation}

Note that in (1)-(4), the boundary conditions are note separated.

The Green's function $g(x,t)$ for (1) is defined as the solution to
\begin{equation}
	\begin{split}
		&Lg = 0\, a<x<t, \, t<x<b; \quad B_1g = 0, \, B_2g = 0, \\
		&g \mbox{ continuous at } x=t, \quad \frac{\partial g}{\partial x}\Big|_{x=t^-}^{x=t^+} = \frac{1}{a_2(t)}.
	\end{split}
\end{equation}

To write down a formula for $g$, we let
\begin{enumerate}[(i)]
	\item $u_i(x,t)$ be the solution to
		\begin{equation}
			Lu_i = 0,\, a<x<t,\, t<x<b; \quad u_i(t,t) = 0, \quad \frac{\partial u_i(t,t)}{\partial x} = \frac{1}{a_2(t)}
		\end{equation}
%
	\item $u_1(x,t)$ be the solution to
		\begin{equation}
			Lu_1 = 0,\, a<x<t,\, t<x<b; \qquad B_1 u_1 = 0,
		\end{equation}
%
	\item $u_2(x,t)$ be the solution to
		\begin{equation}
			Lu_2 = 0,\, a<x<t,\, t<x<b; \qquad B_2 u_2 = 0.
		\end{equation}
\end{enumerate}

Now set
\begin{equation}
	g(x,t) = H(x-t)u_i(x,t) + A_1 u_1(x,t) + A_2 u_2(x,t).
\end{equation}

It can be shown that (9) is the solution to (5). (You do not have to show this.)
Hence (9) is the Green's function for (1).

Show that $A_1$ and $A_2$ can be picked so that (9) satisfies the boundary conditions $B_1g = 0$, $B_2g = 0$.
Hence you are checking part of what one would have to check to show that (9) is the solution to (5)

\ql

\section{}
Construct Green's functions (if they exist) for the following equations.

\subsection{Keener 4.2.3}
\begin{equation*}
	u'' + \alpha^2 u = f(x), u(0) = u(1), u'(0) = u'(1)
\end{equation*}

For what values of $\alpha$ does the Green's function fail to exists?

\ql

\subsection{Keener 4.2.4}
\begin{equation*}
	u'' = f(x), u(0) = 0, \int_0^1 u(x)\,dx = 0
\end{equation*}

\ql

\end{document}
