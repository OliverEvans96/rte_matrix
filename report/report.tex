\documentclass[10pt]{article}

\usepackage{graphicx}
\usepackage{amsmath}
\usepackage{gensymb}
\usepackage{mathtools}
\usepackage{etoolbox}
\usepackage{booktabs}
\usepackage[parfill]{parskip}
\usepackage[numbers]{natbib}
\usepackage{float}
\usepackage{graphicx}
\usepackage{geometry}
\usepackage{multicol}
\usepackage{caption}

\newcommand\mgin{0.5in}
\geometry{
	left=\mgin,
	right=\mgin,
	bottom=\mgin,
	top=\mgin
}

% Set path to import figures from
\graphicspath{{../figures/}}

% Place converted graphics in current directory
\usepackage[outdir=./]{epstopdf}

% Define multicolumn figure-like environment
% from http://tex.stackexchange.com/questions/12262/multicol-and-figures
\newenvironment{mcfig}
	{\par\medskip\noindent\minipage{\linewidth}}
	{\endminipage\par\medskip}

% Define error function for math mode
\newcommand{\erf}{\mbox{erf}}
% Sign function
\newcommand{\sign}{\mbox{sign}}
<<<<<<< Updated upstream
=======
% Natural numbers
\newcommand\N{\mathbb{N}}
% Real numbers
\newcommand\R{\mathbb{R}}
% Complex numbers
\newcommand\C{\mathbb{C}}
% Curly B for basis
\newcommand\BB{\mathcal{B}}
% Curly R for range (not real numbers)
\newcommand\RR{\mathcal{R}}
% Curly N for null space
\newcommand\NN{\mathcal{N}}
% Norm
\newcommand\norm[1]{\left\lVert #1 \right\rVert}
% Uniform Norm
\newcommand\unorm[1]{\left\lVert #1 \right\rVert_\infty}
% Inner Product
\newcommand\ip[1]{\left\langle #1 \right\rangle}
% Absolute value
\newcommand\abs[1]{\left| #1 \right|}
% Complex Conjugate
\newcommand\conj\overline
% Partial derivative
\newcommand\pd[2]{\frac{\partial #1}{\partial #2}}
% Disable paragraph indentation
\setlength{\parindent}{0pt}
% End of proof
\newcommand\qed{\hfill$\blacksquare$\hspace{0.5in}}

% Number this equation
\newcommand\eqnum{\addtocounter{equation}{1}\tag{\theequation}}

>>>>>>> Stashed changes

% arara: pdflatex
% arara: bibtex
% arara: pdflatex
% arara: pdflatex
\begin{document}

%%fakesection Title
\null

\thispagestyle{empty}
\addtocounter{page}{-1}

\begin{center}
    \begin{sffamily}
	\begin{bfseries}
	    \null
	    \vfill
<<<<<<< Updated upstream
	    \Huge{A Kelp Farming Approach to Sustainable Wastewater Nutrient Removal and Bioenergy Production at Wastewater Treatment Plants with Ocean Outfalls} \\
=======
		\Huge{Survey of Solution Techniques for Linear Systems from Finite Difference Methods in 2D Numerical Radiative Transfer}
>>>>>>> Stashed changes

	    \vspace{20pt}
	    \LARGE{Project Summary} \\
		\LARGE{ASSETs to Serve Humanity NSF REU 2016} \\
	    \vspace{20pt}
    \begin{Large}
		Oliver Evans \\
<<<<<<< Updated upstream
                Fred Weiss \\
                Christopher Parker \\
                Emmanuel Arkoh \\[1em]
                Dr. Malena Espa\~nol
=======
		Fred Weiss \\
		Christopher Parker \\
		Emmanuel Arkoh \\[1em]

		Dr. Malena Espa\~nol \\
>>>>>>> Stashed changes
	\vspace{20pt}
	\today
    \end{Large}
	\end{bfseries}
    \end{sffamily}
    \vspace{30pt}

    \null
    \vfill
    \vfill
    \null
\end{center}
\pagebreak


% Increase table cell height (not for header)
\renewcommand{\arraystretch}{1.5}

\begin{multicols}{2}

\section{Introduction}
The impetus for this project was as follows: Dr. Rogers was approached by the operator of a wastewater treatment plant in Boothbay Harbor, Maine, who is facing increasingly demanding EPA regulations limiting the concentration of certain nutrients that is permissible to release into the ocean via wastewater treatment outfalls.
In order to adhere to these stricter requirements using conventional nutrient remediation, a significant quantity of specialized equipment would be necessary, which is not currently present in the Boothbay Harbor plant.
Being surrounded on all sides by water and private property, the treatment plant has no capacity for the additional necessary equipment, and would therefore need to move their entire operations to a new location in order to conform to these new nutrient regulations.
	As an alternative to conventional processing, Dr. Rogers has proposed the cultivation of the macroalgae \textit{Saccharina Latissima} (sugar kelp) near the outfall site.
The purpose of such an undertaking would be twofold: to sequester the nutrients in question and prevent eutrophication of the surrounding ecosystem, and to reduce one of the primary expenses in macroalgae cultivation: nutrient input.
Once grown, a variety of products can be derived from macroalgae, including biofuel, fish/cattle feedstock, and high value chemical materials such as alginate and agar.
Food for human consumption is also a common product of kelp farming, but it may not be ideal for a wastewater treatment application.
Thus, we seek to investigate the feasibility, potential, and optimal implementation of kelp farming in wastewater treatment operations. 
Specifically, I seek to develop a sophisticated model of the light field in a kelp farming environment as a function of both the spacing and depth of the kelp plants, and of the quality of the water itself.
Ole Jacob Broch is a mathematician at SINTEF, a research organization in Trondheim, Norway, who has been working to model the growth of \textit{Saccharina Latissima} via SINMOD, a large-scale 3D hydrodynamical ocean model developed at SINTEF.
My aim is to develop a light model which can be used both independently and in conjunction with Dr. Broch's SINMOD model.

<<<<<<< Updated upstream
=======
We use monochromatic radiative transfer in order to model the light field in an aqueous environment populated by vegetation.
The vegetation (kelp) is modeled by a spatial probability distribution, which we assume to be given.
The two quantities we seek to compute are \textit{radiance} and \textit{irradiance}.
Radiance is the intensity of light in at a particular point in a particular direction, while irradiance is the total light intensity at a point in space, integrated over all angles.
The Radiative Transfer Equation is an integro-partial differential equation for radiance, which has been used primarily in stellar astrophysics; it's application to marine biology is fairly recent \citep{mobley_radiative_2001}.

We study various methods for solving the system of linear equations resulting from discretizing the Radiative Transfer Equation.
In particular, we consider direct methods, stationary iterative methods, and nonstationary iterative methods.
Numerical experiments are performed using Python's \texttt{scipy.sparse} \citep{jones_scipy:_2001} package for sparse linear algebra.
\texttt{IPython} \citep{perez_ipython:_2007} was used for interactive numerical experimentation.

Among those implemented, the nonstationary LGMRES \citep{baker_technique_2005} algorithm is the only algorithm determined to be suitable for this application without further work.
We discuss limitations and potential improvements, including preconditioning, alternative discretization, and reformulation of the RTE.

\subsection{Radiative Transfer}
Let $n$ be the number of spatial dimensions for the problem (i.e., 2 or 3).
Let $x \in \RR^n$.
Let $\Omega$ be the unit sphere in $\RR^n$.
Let $\omega \in \Omega$ be a unit vector in $\RR^n$.
Let $L(x,\omega)$ denote \textit{radiance} position $x$ in the direction $\omega$.
Let $I(x)$ denote \textit{irradiance} at position $x$.
Let $P_k(x)$ be the probability density of kelp at position $x$.
Let $a(x)$ and $b(x)$ denote the absorption and scattering coefficients respectively of the medium, which are both functions of $P_k$.
Let $\beta(\Delta\theta)$ denote the normalized \textit{volume scattering function} or \textit{phase function}, which defines the probability of light scattering at an angle $\Delta\theta$ from it's initial direction in a scattering event.

Then, the Monochromatic Radiative Transfer Equation (RTE) is
\begin{equation}
	\label{eq:rte}
	\begin{aligned}
		\omega \cdot \nabla_x L(x,\omega) &= -(a(x) + b(x)) L(x,\omega) \\
		&\qquad + b \int_\Omega \beta(\omega \cdot \omega') L(x,\omega')\, d\omega'
	\end{aligned}
\end{equation}

Note that in 2 spatial dimensions, this is a 3-dimensional problem ($x,y,\theta$).
Likewise, in 3 spatial dimensions, it is a 5-dimensional problem ($x,y,z,\theta,\phi$).

In this paper, we consider only the 2-dimensional problem, with the hope that sufficiently robust solution techniques for the 2-dimensional problem will be effective in the solution of the 3-dimensional problem, as well.

\subsection{2D Problem}
\label{sec:2d}
We use the downward-pointing coordinate system shown in figure \ref{fig:coords}, measuring $\theta \in [0,2\pi)$ from the positive $x$ axis towards the positive $y$ axis.
Further, we assume that the problem is given on the rescaled spatial domain $[0,1) \times [0,1)$, where $y=0$ is the air-water interface, and $y$ measures depth from the surface.

\begin{figure}[H]
	\centering
	\includegraphics[width=2in]{2d_coords}
	\caption{2D coordinate system}
	\label{fig:coords}
\end{figure}

The 2-dimensional form of \eqref{eq:rte} is given by
\begin{equation}
	\begin{aligned}
		\pd{L}{x} \cos\theta + \pd{L}{y} \sin\theta
		&= -(a+b)L(x,y,\theta) \\
		&+ b\int_0^{2\pi} \beta(\abs{\theta-\theta'})\,d\theta',
	\end{aligned}
	\label{eq:rte2d}
\end{equation}

where $\abs{\theta-\theta'}$ measures the smallest angular difference between $]thet$ and $\theta'$ considering periodicity.

Note that in Cartesian coordinates, there are only spatial, not angular derivatives in the gradient.
In other coordinate systems, this is generally not the case.
	
\subsection{Boundary Conditions}
We assume that the downwelling light from the surface is known, and is defined to be uniform in space by the Dirichlet boundary condition 
\begin{equation}
	L(x,0,\theta) = f(\theta), \quad \mbox{for} \quad \theta \in [0,\pi).
	\label{eq:surf_bc}
\end{equation}

Note that we cannot apply the same idea to upwelling light at the surface, as it cannot be specified from information about the atmospheric light field.
Therefore, we apply the PDE at $y=0$ for $\theta \in [\pi,2\pi)$.

At $y=1$, we assume no upwelling light.
That is,
\begin{equation}
	L(x,0,\theta) = 0, \quad \mbox{for} \quad \theta \in [\pi,2\pi).
	\label{eq:surf_bc}
\end{equation}

As with the upper $y$-boundary, we apply the PDE for $\theta \in [0,\pi)$ so as not to prohibit downwelling light.

In the horizontal direction, we assume periodic boundary conditions.
Assuming that a single discrete group of plants is being simulated, adjusting the width of the domain effectively modifies the spacing between adjacent groups of plants.

\section{Discretization}
In order to solve \eqref{eq:rte2d} numerically, we discretize the spatial derivatives using 2nd order central finite difference approximations (CD2), and we discretize the integral according to the Legendre-Gauss quadrature, as described in chapter 2 of \citet{chandrasekhar_radiative_1960}.
With this in mind, in order to create a spatial-angular grid with $n_x,n_y$, and $n_\theta$ discrete values for $x, y$, and $\theta$ respectively, we use a uniform square spatial discretization with spacing $dx, dy$, and a non-uniform angular discretization according to the roots of the Legendre Polynomial of degree $n_\theta$, denoted $P_{n_\theta}(\theta)$.
In each variable, we discard the uppermost grid point, as indicated by the half-open intervals in the previous sections.

Then, we have the grid
\begin{align}
	x_i &= i\,dx, &\quad i=1,\ldots,n_x \\
	y_j &= j\,dy, &\quad j=1,\ldots,n_y \\
	\theta_k \,\, \mbox{s.t.}\,\, 
	&P_{n_\theta}(\theta_k) = 0, &\quad k=1,\ldots,n_\theta
\end{align}

\subsection{Sparsity Plots}
\begin{figure}[H]
	\centering
	\includegraphics[width=3in]{../img/sparsity/int_kelp1_10x10x16_012.png}
	\caption{Sparsity plot: 10x10x16, ordering 012}
\end{figure}

\begin{figure}[H]
	\centering
	\includegraphics[width=3in]{../img/sparsity/int_kelp1_10x10x16_210.png}
	\caption{Sparsity plot: 10x10x16, ordering 210}
\end{figure}
\subsection{Matrix Properties}
\subsubsection{Diagonal Dominance}
\subsubsection{Spectral Radius}

\section{Direct Methods}
\subsection{Factorizations}
\subsection{Software Packages}

\section{Stationary Iterative Methods}
\subsection{Fixed-Point Iteration}
\subsection{Convergence and Preconditioning}

\section{Nonstationary Iterative Methods}
\subsection{Krylov Subspace Methods}
\subsection{Convergence and Preconditioning}

\section{Numerical Results}

\section{Conclusions}

\nocite{*}
\bibliographystyle{abbrvnat}
\bibliography{rte_matrix_report}
\end{multicols}
>>>>>>> Stashed changes
\end{document}


