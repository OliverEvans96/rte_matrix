\documentclass[10pt]{article}

\usepackage{graphicx}
\usepackage{amsmath,amssymb}
\usepackage{gensymb}
\usepackage{mathtools}
\usepackage{etoolbox}
\usepackage{booktabs}
\usepackage{float}
\usepackage{graphicx}
\usepackage{geometry}
\usepackage{multicol}
\usepackage{caption}

\newcommand\mgin{0.5in}
\geometry{
	left=\mgin,
	right=\mgin,
	bottom=\mgin,
	top=\mgin
}

% Set path to import figures from
\graphicspath{{../figures/}}

% Place converted graphics in current directory
\usepackage[outdir=./]{epstopdf}

% Define multicolumn figure-like environment
% from http://tex.stackexchange.com/questions/12262/multicol-and-figures
\newenvironment{mcfig}
	{\par\medskip\noindent\minipage{\linewidth}}
	{\endminipage\par\medskip}

% Define error function for math mode
\newcommand{\erf}{\mbox{erf}}
% Sign function
\newcommand{\sign}{\mbox{sign}}

% arara: pdflatex
% arara: pdflatex

\begin{document}

%%fakesection Title
\null

\thispagestyle{empty}
\addtocounter{page}{-1}

\begin{center}
    \begin{sffamily}
	\begin{bfseries}
	    \null
	    \vfill
	    \Huge{A Kelp Farming Approach to Sustainable Wastewater Nutrient Removal and Bioenergy Production at Wastewater Treatment Plants with Ocean Outfalls} \\

	    \vspace{20pt}
	    \LARGE{Project Summary} \\
		\LARGE{ASSETs to Serve Humanity NSF REU 2016} \\
	    \vspace{20pt}
    \begin{Large}
		Oliver Evans \\
                Fred Weiss \\
                Christopher Parker \\
                Emmanuel Arkoh \\[1em]
                Dr. Malena Espa\~nol
	\vspace{20pt}
	\today
    \end{Large}
	\end{bfseries}
    \end{sffamily}
    \vspace{30pt}

    \null
    \vfill
    \vfill
    \null
\end{center}
\pagebreak


% Increase table cell height (not for header)
\renewcommand{\arraystretch}{1.5}

\begin{multicols}{2}

\section{Introduction}
The impetus for this project was as follows: Dr. Rogers was approached by the operator of a wastewater treatment plant in Boothbay Harbor, Maine, who is facing increasingly demanding EPA regulations limiting the concentration of certain nutrients that is permissible to release into the ocean via wastewater treatment outfalls.
In order to adhere to these stricter requirements using conventional nutrient remediation, a significant quantity of specialized equipment would be necessary, which is not currently present in the Boothbay Harbor plant.
Being surrounded on all sides by water and private property, the treatment plant has no capacity for the additional necessary equipment, and would therefore need to move their entire operations to a new location in order to conform to these new nutrient regulations.
	As an alternative to conventional processing, Dr. Rogers has proposed the cultivation of the macroalgae \textit{Saccharina Latissima} (sugar kelp) near the outfall site.
The purpose of such an undertaking would be twofold: to sequester the nutrients in question and prevent eutrophication of the surrounding ecosystem, and to reduce one of the primary expenses in macroalgae cultivation: nutrient input.
Once grown, a variety of products can be derived from macroalgae, including biofuel, fish/cattle feedstock, and high value chemical materials such as alginate and agar.
Food for human consumption is also a common product of kelp farming, but it may not be ideal for a wastewater treatment application.
Thus, we seek to investigate the feasibility, potential, and optimal implementation of kelp farming in wastewater treatment operations. 
Specifically, I seek to develop a sophisticated model of the light field in a kelp farming environment as a function of both the spacing and depth of the kelp plants, and of the quality of the water itself.
Ole Jacob Broch is a mathematician at SINTEF, a research organization in Trondheim, Norway, who has been working to model the growth of \textit{Saccharina Latissima} via SINMOD, a large-scale 3D hydrodynamical ocean model developed at SINTEF.
My aim is to develop a light model which can be used both independently and in conjunction with Dr. Broch's SINMOD model.

\end{document}


